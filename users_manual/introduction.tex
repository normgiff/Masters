\section{Introduction and Motivation}
At the University of Utah, students enrolled in Digital VLSI Design (CS/ECE 5710/6710) work in teams to design an integrated circuit (IC) for their final project. Optionally, students may send their chip designs to MOSIS, an IC foundry, for fabrication. As per MOSIS's rules, students who fabricate their chip designs are required to test their received chips and report their results back to MOSIS. 

At least one student from any team that chooses to fabricate their chip must enroll in CS/ECE 6712, which teaches students how to test their chips using a device called an application-specific integrated circuit (ASIC) tester. Presently, the University of Utah has two ASIC testers: a Verigy 93000 and a Tektronix LV500. As of this writing, both machines are in troubled states. The Verigy 93000 is no longer operational due to a hard drive failure. The LV500 has many failed sectors, so the system as a whole may prove unreliable when testing a chip. 

Simply purchasing a new ASIC tester is not feasible. ASIC testers are extremely expensive machines (millions to billions of dollars) --  partly because of a niche market, but mainly because they are incredibly difficult systems to engineer, construct, and maintain. 

Our project, which we term the LV600, is an ASIC tester based largely upon the LV500. The project's intention is to provide an ASIC tester that is good enough for most student-designed VLSI chips. This document serves as both the final report for our project and the user's manual for the system that we hope to ultimately provide to the University. 

In this document, we also include ideas for how to build upon our project, and we strongly encourage students to consider building an ASIC tester for their senior project if they have any interest in such machines. The project provides so many learning experiences -- it's conceptually simple, but the devil lies in the details. 

\newpage